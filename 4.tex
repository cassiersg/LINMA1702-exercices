\section{Méthode du simplexe}

\begin{enumerate}

  \item Soit le problème d'optimisation

    $
    \begin{array}{lrcr}
      \mini & -2x_1 - x_2 \\
      & x_1 - x_2 & \leq & 2\\
      & x_1 +  x_2 & \leq & 6\\
      & x_1, x_2 & \geq & 0
    \end{array}
    $

    Convertissez ce problème sous forme standard et trouvez un sommet pour
    lequel $x_1 = x_2 =0$.
    Résolvez le problème au moyen de la méthode
    du simplexe.
    Tracez une représentation graphique en terme des variables
    $x_1, x_2$ et indiquez le chemin suivi par la méthode.

    \begin{solution}
      En introduisant les variables d'écart $x_3$ et $x_4$,
      le tableau simplexe devient
      \[
        \begin{array}{cccc|l}
          -2 & -1 & 0 & 0 & z\\
          \hline
           1 & -1 & 1 & 0 & 2\\
           1 &  1 & 0 & 1 & 6
        \end{array}
      \]
      Le sommet $(0,0,2,6)$ n'est pas optimal car les coûts réduits
      $\tilde{c}_1 = -2$ et $\tilde{c}_2 = -1$ sont négatifs.
      Faisons rentrer $x_1$ dans la base (bien qu'on aurait tout autant
      pu faire rentrer $x_2$).
      Regardons qui on doit sortir de la base pour garder un sommet
      admissible.
      Si $x_1 = \lambda > 0$, on a $x = (\lambda, 0, 2-\lambda,6-\lambda)$.
      Si on voulait faire sortir, $x_4$, on aurait eu $x_4=0$ et
      donc $\lambda = 6$ et $x_3 = -4$ ce qui n'est pas admissible.
      C'est donc $x_3$ qui sort de la base.
      Le tableau simplexe devient alors
      \[
        \begin{array}{cccc|l}
          0 & -3 &  2 & 0 & z+4\\
          \hline
          1 & -1 &  1 & 0 & 2\\
          0 &  2 & -1 & 1 & 4
        \end{array}
      \]
      Le sommet $(2,0,0,4)$ n'est pas non plus optimal car le coût
      réduit $\tilde{c}_2 = -1$ est négatif.
      Faisons donc rentrer $x_2$ dans la base car c'est lui qui a un
      coût réduit négatif.
      À nouveau, on a $x = (2+\lambda,\lambda,0,4-2\lambda)$.
      Comme $\lambda > 0$, $x_1$ ne peut pas être nul.
      C'est donc $x_4$ qui sort de la base.
      \[
        \begin{array}{cccc|l}
          0 & 0 &  1/2 & 3/2 & z+10\\
          \hline
          1 & 0 &  1/2 & 1/2 & 4\\
          0 & 1 & -1/2 & 1/2 & 2
        \end{array}
      \]
      Les coût réduits $\tilde{c}_3$ et $\tilde{c}_4$ sont négatifs,
      le sommet $(4,2,0,0)$ est donc optimal.
      On a donc $\xopt = (4,2,0,0)$ avec $z^* = -10$.
    \end{solution}

  \item Considérez le problème

    $
    \begin{array}{llcr}
      \mini & 20 x_1+\alpha x_2+12 x_3\\
      & x_1  & \leq & 400\\
      &2x_1+\beta x_2+x_3 & \leq & 1000\\
      &2x_1+\gamma x_2+ 3x_3 & \leq & 1600\\
      & x_1, x_2, x_3 & \geq & 0
    \end{array}
    $

    Proposez, si possible,
    des valeurs pour $\alpha, \beta$ et $\gamma$ pour lesquelles:

    \begin{enumerate}

      \item Le coût optimal est fini et la solution optimale est unique.

      \item Le coût optimal est fini
        et il y a une infinité de solutions optimales.

      \item Le coût optimal est non borné
        (trouvez une paramétrisation de valeurs de $x$
        parmi lesquelles se trouvent des solutions de coûts
        arbitrairement faibles).

      \item Le poly\`edre possède un sommet dégénéré.

    \end{enumerate}

    \begin{solution}
      En introduisant les variables d'écart $x_4$, $x_5$ et $x_6$,
      le tableau simplexe devient
      \[
        \begin{array}{cccccc|l}
          20 & \alpha & 12 & 0 & 0 & 0 & z\\
          \hline
           1 & 0      &  0 & 1 & 0 & 0 & 400\\
           2 & \beta  &  1 & 0 & 1 & 0 & 1000\\
           2 & \gamma &  3 & 0 & 0 & 1 & 1600
        \end{array}
      \]
      \begin{enumerate}
        \item Si $\alpha > 0$, $(0,0,0)$ est l'unique sommet optimal.
          En effet, il est impossible d'avoir $z^* = 0$ avec d'autres sommets.
        \item Si $\alpha = 0$, $(0,x_2,0)$ est optimal pour tout $x_2$
          tel que la solution est admissible.
        \item Si $\alpha < 0$ et $\beta,\gamma \leq 0$.
          $(0,\lambda,0)$ est admissible pour tout $\lambda > 0$ et
          $z^*$ est aussi petit que l'on veut.
        \item Pour avoir un sommet dégénéré, il faut qu'un terme indépendant
          vaille 0.
          En sortant $x_2$ de la base et en y fesant rentrer $x_3$, on
          optient le tableau simplexe suivant (si $\beta \neq 0$)
          \[
            \begin{array}{cccccc|l}
              20 & \alpha & 12 & 0 & 0 & 0 & z\\
              \hline
              1 & 0      &  0 & 1 & 0 & 0 & 400\\
              2 & \beta  &  1 & 0 & 1 & 0 & 1000\\
              2-2\frac{\gamma}{\beta} & 0
              & 3-\frac{\gamma}{\beta} & 0 & -\frac{\gamma}{\beta} & 1
              & 1600 - 1000\frac{\gamma}{\beta}
            \end{array}
          \]
          Un choix possible est donc $(\beta,\gamma) = (1000,1600)$
          qui aurait comme sommet dégénéré $(0,1,0)$.
      \end{enumerate}
    \end{solution}

  \item
    Résoudre par l'algorithme du simplexe les problèmes
    \begin{enumerate}
      \item
        $
        \begin{array}{llcr}
          \maxi & 2x_1+3x_2\\
                & x_1+2x_2 & \leq & 4\\
                & x_1+x_2 & = & 3\\
                & x_1, x_2& \geq & 0
        \end{array}
        $
      \item
        $
        \begin{array}{llcr}
          \maxi & 20 x_1+16x_2+12 x_3\\
                & x_1  & \leq & 400\\
                &2x_1+x_2+x_3 & \leq & 1000\\
                &2x_1+2x_2+ 3x_3 & \leq & 1600\\
                & x_1, x_2, x_3 & \geq & 0
        \end{array}
        $
      \item
        $
        \begin{array}{llcr}
          \maxi & 4x_1+3x_2+6x_3\\
                & 3x_1+x_2+3x_3 & \leq & 30\\
                & 2x_1+2x_2+3x_3 & \leq & 40\\
                & x_1, x_2, x_3 & \geq & 0
        \end{array}
        $
      \item
        $
        \begin{array}{llcr}
          \maxi & -x_1+4x_2\\
                & -3x_1+4x_2 & \leq & 6\\
                & x_1+2x_2 & \leq & 4\\
                & x_2& \geq & -3
        \end{array}
        $
    \end{enumerate}

    \begin{solution}
      \begin{enumerate}
        \item En mettant le problème sous forme standard,
          on obtient le tableau simplexe suivant
          \[
            \begin{array}{ccc|l}
              -2 & -3 & 0 & -z\\
              \hline
              0 &  1 & 1 & 1\\
              1 &  1 & 0 & 3\\
            \end{array}
          \]
          on peut facilement trouver un sommet sans devoir passer
          par le problème annexe.
          On trouve
          \[
            \begin{array}{ccc|l}
              0 & -1 & 0 & -z+6\\
              \hline
              0 &  1 & 1 & 1\\
              1 &  1 & 0 & 3\\
            \end{array}
          \]
          En faisant rentrer $x_2$ dans la base,
          on doit faire sortir $x_3$.
          On a alors
          \[
            \begin{array}
              {ccc|l}
              0 & -1 & 0 & -z+7\\
              \hline
              0 &  1 & 1 & 1\\
              1 &  1 & 0 & 3\\
            \end{array}
          \]
          Dès lors, $\xopt = (2,1,0)$ et $z^* = 7$.
        \item
          \label{itm:sim400}
          Le tableau simplexe est le suivant
          \[
            \begin{array}{cccccc|l}
              -20 & -16 & -12 & 0 & 0 & 0 & -z\\
              \hline
              1 &   0 &   0 & 1 & 0 & 0 & 400\\
              2 &   1 &   1 & 0 & 1 & 0 & 1000\\
              2 &   2 &   3 & 0 & 0 & 1 & 1600
            \end{array}
          \]

          En faisant renter $x_1$ dans la base,
          on voit qu'on doit faire sortir $x_4$.
          On a donc
          \[
            \begin{array}{cccccc|l}
              0 & -16 & -12 & 20 & 0 & 0 & -z+8000\\
              \hline
              1 &   0 &   0 &  1 & 0 & 0 & 400\\
              0 &   1 &   1 & -2 & 1 & 0 & 200\\
              0 &   2 &   3 & -2 & 0 & 1 & 800
            \end{array}
          \]

          Faisons maintenant rentrer $x_2$ dans la base en faisant sortir $x_5$
          \[
            \begin{array}{cccccc|l}
              0 & 0 & 4 & -12 & 16 & 0 & -z+11200\\
              \hline
              1 & 0 & 0 &   1 &  0 & 0 & 400\\
              0 & 1 & 1 &  -2 &  1 & 0 & 200\\
              0 & 0 & 1 &   2 & -2 & 1 & 400
            \end{array}
          \]

          Essayons maintenant en faisant rentre $x_4$.
          On doit alors sortir $x_6$.
          \[
            \begin{array}{cccccc|l}
              0 & 0 &  10  & 0 &  4 &  6   & -z+13600\\
              \hline
              1 & 0 & -1/2 & 0 &  1 & -1/2 & 200\\
              0 & 1 &  2   & 0 & -1 &  1   & 600\\
              0 & 0 &  1/2 & 1 & -1 &  1/2 & 200
            \end{array}
          \]
          La solution optimale est donc $x^* = (200,600,0)$ avec $z^* = 13600$.
        \item
          On a le tableau simplexe suivant
          \[
            \begin{array}{ccccc|l}
              -4 & -3 & -6 & 0 & 0 & -z\\
              \hline
               3 &  1 &  3 & 1 & 0 & 30\\
               2 &  2 &  3 & 0 & 1 & 40\\
            \end{array}
          \]

          En faisant rentrer $x_3$ dans la base,
          $x_4$ doit sortir, ce qui donne
          \[
            \begin{array}{ccccc|l}
               2 & -1   & 0 & 2   & 0 & -z+60\\
              \hline
               1 &  1/3 & 1 & 1/3 & 0 & 10\\
              -1 &  1   & 0 & -1  & 1 & 10\\
            \end{array}
          \]

          Faisons maintenant rentrer $x_2$ et donc sortir $x_5$
          \[
            \begin{array}{ccccc|l}
               1   & 0 & 0 &  1   &  1   & -z+70\\
              \hline
               4/3 & 0 & 1 &  2/3 & -1/3 & 20/3\\
              -1   & 1 & 0 & -1   &  1   & 10\\
            \end{array}
          \]
          d'où la solution optimale $\xopt = (0,10,\frac{20}{3})$
          de coût optimal $z^* = 70$.
        \item
          Faisons les changements de variables $x_1 = x_3 - x_4$ et
          $x_2 = x_5-3$.
          Le problème devient alors
          \begin{align*}
            -12 - \min x_3 - x_4 - 4x_5\\
            -3x_3 + 3x_4 + 4x_5 & \leq 18\\
            x_3 - x_4 + 2x_5 & \leq 10\\
            x_3,x_4,x_5 & \geq 0.
          \end{align*}
          En ajoutant les variables d'écart nécessaire,
          on a le tableau simplexe
          \[
            \begin{array}{ccccc|l}
              1 & -1 & -4 & 0 & 0 & -z-12\\
              \hline
              -3 & 3 & 4 & 1 & 0 & 18\\
              1 & -1 & 2 & 0 & 1 & 10\\
            \end{array}
          \]

          En ajoutant $x_5$ à la base, on doit retirer $x_6$.
          On a alors
          \[
            \begin{array}{ccccc|l}
              -2 & 2 & 0 & 1 & 0 & -z+6\\
              \hline
              -3/4 & 3/4 & 1 & 1/4 & 0 & 9/2\\
              5/2 & -5/2 & 0 & -1/2 & 1 & 1\\
            \end{array}
          \]

          Ajoutons maintenant $x_3$ à la base. $x_7$ doit alors sortir
          \[
            \begin{array}{ccccc|l}
              0 &  0 & 0 &  3/5  & 4/5  & -z+34/5\\
              \hline
              0 &  0 & 1 &  1/10 & 3/10 & 24/5\\
              1 & -1 & 0 & -1/5  & 2/5  & 2/5\\
            \end{array}
          \]

          On a donc $(x_3,x_4,x_5) = \frac{1}{5}(2,0,24)$ d'où
          $\xopt = \frac{1}{5}(2, 9)$ avec un coût optimal
          $z^* = \frac{34}{5}$.
      \end{enumerate}
    \end{solution}

  \item Nous reprenons un des problèmes précédents.
    Un étudiant dispose de 100 heures de travail pour
    étudier les examens A, B et C.
    Il pense gagner par heure de travail sur chaque cours 1/5 de points pour
    le cours A, 2/5 de points pour le cours B et 3/5 pour le cours C.
    Chaque examen est coté sur 20.
    Les exercices de ces cours comptent pour la moitié de la cote finale.
    Ses résultats pour les exercices lui ont été communiqués.
    Il a obtenu 12/20 pour A, 12.5/20 pour B et 13.4/20 pour C.
    L'étudiant doit obtenir au minimum une cote globale de 10/20 pour chaque
    cours.
    Tous les cours ont la même pondération et l'étudiant désire obtenir la
    moyenne la plus élevée possible.

    Formulez ce problème comme un problème d'optimisation linéaire  et résolvez-le.
    Vous pouvez utiliser le fait que l'étudiant a avantage à utiliser la
    totalité des 100 heures de travail.
    L'étudiant obtiendra-t-il une distinction?

    \begin{solution}
      Soient $x_A$, $x_B$ et $x_C$, respectivement
      le nombre d'heures étudiées pour $A$, $B$ et $C$.
      Le nombres d'heures optimaux sont les solutions du problème
      d'optimisation suivant
      \begin{align*}
        \max x_A + 2x_B + 3x_C\\
        x_A & \leq 100\\
        x_A & \geq 40\\
        x_B & \leq 50\\
        x_B & \geq 18.75\\
        x_C & \leq 100/3\\
        x_C & \geq 11\\
        x_A + x_B + x_C & = 100
      \end{align*}
      En posant $x_1,x_2,x_3$ le nombre d'heures étudiées pour faire plus
      que 10, c'est à dire
      \begin{align*}
        x_A & = 40 + x_1\\
        x_B & = 18.75 + x_2\\
        x_C & = 11 + x_3
      \end{align*}

      On sait que les $x_1,x_2,x_3$ donnant les $x_A,x_B,x_C$ optimaux
      sont solutions du problème d'optimisation suivant.
      \begin{align*}
        \max x_1 + 2x_2 + 3x_3\\
        x_A & \leq 60\\
        x_B & \leq 31.25\\
        x_C & \leq 67/3\\
        x_A + x_B + x_C & = 30.25\\
        x & \geq 0
      \end{align*}

      Le tableau simplexe est alors
      \[
        \begin{array}{cccccc|l}
          -1 & -2 & -3 & 0 & 0 & 0 & -z\\
          \hline
          1 & 0 & 0 & 1 & 0 & 0 & 60\\
          0 & 1 & 0 & 0 & 1 & 0 & 31.25\\
          0 & 0 & 1 & 0 & 0 & 1 & 67/3\\
          1 & 1 & 1 & 0 & 0 & 0 & 30.25
        \end{array}
      \]

      Il y a 4 contraintes mais que 3 variables de base, il en manque donc
      une pour faire un sommet.
      On peut cependant facilement ajouter $x_1$ dans la base
      car $61/2 < 60$ ou $x_2$ car $61/2 < 31.25$ (en effet,
      si on essaie de faire rentrer $x_3$, on doit faire sortir $x_6$
      alors que $x_1$ et $x_2$ ne font sortir personne).
      Il n'y a donc pas besoin de passer par le problème annexe pour trouver
      un sommet de départ.
      Faisons rentrer $x_2$
      \[
        \begin{array}{cccccc|l}
           1 & 0 & -1 & 0 & 0 & 0 & -z + 61\\
          \hline
           1 & 0 &  0 & 1 & 0 & 0 & 60\\
          -1 & 0 & -1 & 0 & 1 & 0 & 1\\
           0 & 0 &  1 & 0 & 0 & 1 & 67/3\\
           1 & 1 &  1 & 0 & 0 & 0 & 30.25
        \end{array}
      \]

      Faisons maintenant rentrer $x_3$, pour cela, on doit faire sortir $x_6$
      \[
        \begin{array}{cccccc|l}
           1 & 0 & 0 & 0 & 0 &  1 & -z + 250/3\\
          \hline
           1 & 0 & 0 & 1 & 0 &  0 & 60\\
          -1 & 0 & 0 & 0 & 1 &  1 & 70/3\\
           0 & 0 & 1 & 0 & 0 &  1 & 67/3\\
           1 & 1 & 0 & 0 & 0 & -1 & 95/12
        \end{array}
      \]

      On trouve donc $(x_1,x_2,x_3) = (0,95/12,67/3)$ d'où
      \[ (x_A^*,x_B^*,x_C^*) = (40,80/3,100/3). \]
      Il aura donc $10$ pour A, $11.58\bar{3}$ pour B et $20$ pour C
      ce qui lui fait une moyenne de $13.86\bar{1}$ ou encore
      $69.30\bar{5}\,\%$.
      Il aura donc une distinction car la barre est à $68\,\%$.
    \end{solution}

  \item Une société produit des biens A, B et C.
    La production des biens nécessite l'utilisation de 4 machines.
    Les temps de production et les profits générés sont repris dans le tableau

    $
    \begin{array}{l|llll|l}
      & 1 & 2 & 3 & 4 & \mbox{profit}\\
      \hline
      A & 1 & 3 & 1 & 2 & 6\\
      B & 6 & 1 & 3 & 3 & 6\\
      C & 3 & 3 & 2 & 4 & 6
    \end{array}
    $
    \\

    Si les temps de production disponibles sur les machines 1, 2, 3 et 4  sont de 84, 42, 21 et 42,
    déterminez la quantité de biens à produire pour maximiser le profit.

    \begin{solution}
      Soit $x_i$ la quantité de biens $i$ produite.
      Le problème d'optimisation est donc le suivant
      \begin{align*}
        -\min -x_A - x_B - x_C\\
         x_A + 6x_B + 3x_C & \leq 84\\
        3x_A +  x_B + 3x_C & \leq 42\\
         x_A + 3x_B + 2x_C & \leq 21\\
        2x_A + 3x_B + 4x_C & \leq 42\\
        x & \geq 0.
      \end{align*}
      En introduisant les variables d'écart nécessaire,
      on obtient le tableau simplexe suivant
      \[
        \begin{array}{ccccccc|l}
          -6 & -6 & -6 & 0 & 0 & 0 & 0 & -z\\
          \hline
           1 &  6 &  3 & 1 & 0 & 0 & 0 & 84\\
           3 &  1 &  3 & 0 & 1 & 0 & 0 & 42\\
           1 &  3 &  2 & 0 & 0 & 1 & 0 & 21\\
           2 &  3 &  4 & 0 & 0 & 0 & 1 & 42\\
        \end{array}
      \]
      En faisant rentrer $x_1$, on obtient
      \[
        \begin{array}{ccccccc|l}
          0 & -4   & 0 & 0 &  2   & 0 & 0 & -z+84\\
          \hline
          0 & 17/3 & 2 & 1 & -1/3 & 0 & 0 & 70\\
          1 &  1/3 & 1 & 0 &  1/3 & 0 & 0 & 14\\
          0 &  8/3 & 1 & 0 & -1/3 & 1 & 0 & 7\\
          0 &  7/3 & 2 & 0 & -2/3 & 0 & 1 & 14\\
        \end{array}
      \]
      Rentrons maintenant $x_2$ dans la base pour obtenir
      \[
        \begin{array}{ccccccc|l}
          0 & 0 &  3/2 & 0 &  3/2  &   3/2 & 0 & -z+189/2\\
          \hline
          0 & 0 & -1/8 & 1 &  9/24 & -17/8 & 0 & 441/8\\
          1 & 0 &  7/8 & 0 &  9/24 &  -3/8 & 0 & 105/8\\
          0 & 1 &  3/8 & 0 & -1/8  &   3/8 & 0 & 21/8\\
          0 & 0 &  9/8 & 0 & -9/24 &  -7/8 & 1 & 63/8\\
        \end{array}
      \]
      Le sommet optimal est donc $x^* = \frac{1}{8}(105,21,0)$ avec comme
      coût optimal $z^* = 189/2$.
    \end{solution}

  \item Résoudre par la méthode du simplexe en utilisant la règle de Bland

    $
    \begin{array}{llcr}
      \maxi & 10 x_1-57 x_2-9x_3-24x_4\\
      & 0.5x_1-5.5x_2-2.5x_3+9x_4 & \leq & 0\\
      & 0.5x_1-1.5x_2-0.5x_3+x_4 & \leq & 0\\
      & x_1 & \leq &1\\
      & x_1, x_2, x_3, x_4 & \geq & 0
    \end{array}
    $


    \begin{solution}
      La règle de Bland consiste à choisir la variable $x_r$ de plus petit
      indice $r$ parmi les variables candidates à l'entrée $i = 1,2, \dots, n$
      (idem pour les variables candidates à la sortie).

      Ça permet d'éviter qu'on boucle infiniment sur des sommets
      dégénérés de même coût dans l'algorithme du simplexe.

      On a donc
      \[
        \begin{array}{ccccccc|l}
          -10 & 57   & 9    & 24 & 0 & 0 & 0 & -z\\
          \hline
          0.5 & -5.5 & -2.5 &  9 & 1 & 0 & 0 & 0\\
          0.5 & -1.5 & -0.5 &  1 & 0 & 1 & 0 & 0\\
          1   &  0   &  0   &  0 & 0 & 0 & 1 & 1
        \end{array}
      \]

      Faisons rentrer $x_1$ dans la base.
      On a le choix entre faire sortir $x_5$ et $x_6$.
      La règle de Bland nous impose de faire sortir $x_5$.
      \[
        \begin{array}{ccccccc|l}
          0 & -53 & -41 & 204 & 20 & 0 & 0 & -z\\
          \hline
          1 & -11 &  -5 &  18 &  2 & 0 & 0 & 0\\
          0 &   4 &   2 &  -8 &  0 & 1 & 0 & 0\\
          0 &  11 &   5 & -18 & -2 & 0 & 1 & 1
        \end{array}
      \]

      Faisons maintenant rentrer $x_2$.
      On doit donc faire sortir $x_6$.
      \[
        \begin{array}{ccccccc|l}
          0 & 0 & -29/2 & 98 & 20 &  53/4 & 0 & -z\\
          \hline
          1 & 0 &   1/2 & -4 &  2 &  11/4 & 0 & 0\\
          0 & 1 &   1/2 & -2 &  0 &   1/4 & 0 & 0\\
          0 & 0 &  -1/2 &  4 & -2 & -11/4 & 1 & 1
        \end{array}
      \]

      On doit maintenant faire rentrer $x_3$.
      On a le choix entre faire sortir $x_1$ ou $x_2$ mais la
      règle de Bland nous impose de sortir $x_1$.
      \[
        \begin{array}{ccccccc|l}
          29 & 0 & 0 & -18 & 78 & 93   & 0 & -z\\
          \hline
           1 & 0 & 1 &  -8 &  4 & 11/4 & 0 & 0\\
          -1 & 1 & 0 &   2 & -2 & -5/4 & 0 & 0\\
           1 & 0 & 0 &   0 &  0 &  0   & 1 & 1
        \end{array}
      \]

      On doit alors faire rentrer $x_4$ et pour cela faire sortir
      $x_2$.
      \[
        \begin{array}{ccccccc|l}
          20   & 9   & 0 & 0 & 60   & 141/2 & 0 & -z\\
          \hline
          -3   & 4   & 1 & 0 & -4   & -9/2  & 0 & 0\\
          -1/2 & 1/2 & 0 & 1 & -1/2 & -5/4  & 0 & 0\\
           1   & 0   & 0 & 0 &  0   &  0    & 1 & 1
        \end{array}
      \]

      $(0,0,0,0)$ est donc assurément un sommet optimal.
      Mais comme le coût n'a jamais changé, tous les sommets
      étaient optimaux.

    \end{solution}

  \item Proposez une méthode de recherche d'un sommet du polyhèdre

    $
    \begin{array}{lrcr}
      & 2x_1-3x_2 +2x_3 & \geq & 3\\
      & -x_1+x_2 +x_3 & \geq & 5\\
      & x_1, x_2, x_3 & \geq & 0
    \end{array}
    $

    \begin{solution}
      Commençons par écrire le polyèdre sous forme standard

      \begin{align*}
        2x_1 - 3x_2 + 2x_3 - x_4 & = 3\\
        -x_1 +  x_2 +  x_3 - x_5 & = 5\\
         x_1, x_2, x_3, x_4, x_5 & \geq 0
      \end{align*}

      Il n'est pas facile de deviner un sommet immédiatement car en
      prenant comme base $x_4,x_5$, on arrive à $(x_4,x_5)=(-3,-5)$
      qui n'est pas admissible.

      On fait donc appel au problème annexe
      \begin{align*}
        \min \sum_i y_i\\
        Ax + y & = b\\
        x,y & \geq 0.
      \end{align*}

      Si le coût optimal du problème annexe est nul,
      c'est que le polyèdre ne possède pas de sommet.
      Sinon, $(x_1^*, x_2^*, x_3^*)$ forme un sommet optimal.
      On résoud donc le problème annexe avec le simplexe.
      Le tableau simplexe est

      \[
        \begin{array}{ccccccc|l}
           0 &  0 & 0 &  0 &  0 & 1 & 1 & z\\
          \hline
           2 & -3 & 2 & -1 &  0 & 1 & 0 & 3\\
          -1 &  1 & 1 &  0 & -1 & 0 & 1 & 5\\
        \end{array}
      \]

      L'avantage du problème annexe c'est qu'on peut trouver immédiatement
      un sommet pour commencer l'algorithme du simplexe
      si on avait $b \geq 0$.
      Il suffit de prendre $y$ comme base.
      En effet, $x = (0,0,0,0,0,5,5)$ est un sommet de ce polyèdre annexe.
      Bien qu'on ne demande qu'une méthode de recherche,
      notre curiosité nous impose de continuer la résolution,
      on obtient alors le tableau simplexe

      \[
        \begin{array}{ccccccc|l}
          -1 &  2 & -3 &  1 &  1 & 0 & 0 & z-8\\
          \hline
           2 & -3 &  2 & -1 &  0 & 1 & 0 & 3\\
          -1 &  1 &  1 &  0 & -1 & 0 & 1 & 5\\
        \end{array}
      \]

      On peut faire rentrer $x_3$ dans la base,
      on obtient alors

      \[
        \begin{array}{ccccccc|l}
           2 & -5/2 &  0 & -1/2 &  1 &  3/2 & 0 & z-7/2\\
          \hline
           1 & -3/2 &  1 & -1/2 &  0 &  1/2 & 0 & 3/2\\
          -2 &  5/2 &  0 &  1/2 & -1 & -1/2 & 1 & 7/2\\
        \end{array}
      \]

      On voit qu'on s'approche de $z = 0$ mais on y est pas encore.
      D'ailleurs il reste des coûts réduits négatifs, faisons maintenant
      rentrer $x_2$ dans la base, on obtient

      \[
        \begin{array}{ccccccc|l}
           0   &  0   &  0 &  0   &  0   &  1   & 1   & z\\
          \hline
          -1/5 &  0   &  1 & -1/5 & -3/5 &  1/5 & 3/5 & 18/5\\
          -4/5 &  1   &  0 &  1/5 & -2/5 & -1/5 & 2/5 &  7/5\\
        \end{array}
      \]

      On a maintanenant bien $z = 0$.
      Notre polyèdre avait donc au moins un sommet.
      $\frac{1}{5}(0,7,18)$ est un de ces sommets.
    \end{solution}

  \item  Considérez le problème d'optimisation

    $
    \begin{array}{llcr}
      \mini & 20 x_1+\alpha x_2+12 x_3\\
      & x_1  & \leq & 4\\
      &2x_1 - x_2+x_3 & \leq & 10\\
      &2x_1+\beta x_2+ 3x_3 & \leq & 16\\
      & x_1, x_2, x_3 & \geq & 0
    \end{array}
    $

    Trouvez une solution optimale au moyen de l'algorithme du simplexe
    lorsque $\alpha = -2$ et $\beta = 1$.
    Proposez des valeurs pour $\alpha$ et $\beta$ pour lesquelles
    le coût optimal est non borné et
    proposez dans ce cas une solution pour laquelle le coût
    optimal est inférieur à $-1000$.
    % FIXME: ne serait-ce pas mieux de dire "solution ADMISSIBLE"
    %        car solution est défini comme n'importe quel x

    \begin{solution}
      Pour $\alpha = -2$ et $\beta = 1$, le tableau simplexe est
      \[
        \begin{array}{cccccc|l}
          20 & -2 & 12 & 0 & 0 & 0 & z\\
          \hline
          1  &  0 &  0 & 1 & 0 & 0 & 4\\
          2  & -1 &  1 & 0 & 1 & 0 & 10\\
          2  &  1 &  3 & 0 & 0 & 1 & 16\\
        \end{array}
      \]
      en ajoutant $x_2$ à la base,
      il faut retirer $x_6$, on a alors
      \[
        \begin{array}{cccccc|l}
          24 & 0 & 18 & 0 & 0 & 2 & z+32\\
          \hline
          1  & 0 &  0 & 1 & 0 & 0 & 4\\
          4  & 0 &  4 & 0 & 1 & 1 & 26\\
          2  & 1 &  3 & 0 & 0 & 1 & 16\\
        \end{array}
      \]
      La solution optimale vaut donc $\xopt = (0,16,0)$ avec
      un coût optimal de $-32$.

      Pour que le coût soit non-borné, il suffit que $\alpha < 0$ et que
      $\beta \leq 0$.
      Prenons $(\alpha,\beta) = (-1,-1)$.
      On remarque alors que $(0,1001,0)$ est admissible et que son
      coût $-1001$ est inférieur à $-1000$.
    \end{solution}

  \item  Considérez le problème d'optimisation

    $
    \begin{array}{llcr}
      \maxi
      & \alpha x_1 + 16x_2+ 12x_3\\
      &        x_1                & \leq & 400\\
      &       2x_1 +   x_2+   x_3 & \leq & 1000\\
      &       2x_1 +  2x_2+  3x_3 & \leq & 1600\\
      &        x_1, x_2, x_3 & \geq & 0
    \end{array}
    $

    Trouvez une solution optimale au moyen de l'algorithme du simplexe
    lorsque $\alpha=20$.
    Existe-t-il une valeur de $\alpha$ pour laquelle le coût optimal
    est non borné?
    Si oui,
    proposez une solution pour laquelle le coût est supérieur à 10000.
    Si non, justifiez votre réponse.

    \begin{solution}
      Pour la résolution du simplexe, voir l'exercice~\ref{itm:sim400}.

      Si le coût optimal est non borné,
      c'est qu'il existe une demi-droite incluse au domaine admissible
      au long de laquelle $x_1 + 16x_2 + 12x_3$ augmente.
      Seulement, on peut remarqué géométriquement que le domaine admissible
      est compacte.
      On peut aussi s'en sortir algébriquement en se disant que
      si le coût augmente à l'infini, c'est au moins une variable augmente
      à l'infini.
      En tenant compte des contraintes, comme tous les coefficients
      sont positifs, il faut qu'au moins une variable tende vers
      $-\infty$ ce qui est impossible car elles sont positive.
      Tout cela pour dire qu'il n'existe pas de valeur d'$\alpha$
      pour laquelle le coût optimal est non borné.
    \end{solution}

\end{enumerate}
